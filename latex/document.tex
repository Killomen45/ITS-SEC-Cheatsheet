\documentclass[]{report}


% Title Page
\title{ICT Sec Linux Cheatsheet}
\author{Varia gente}
\renewcommand{\chaptername}{Capitolo}






\begin{document}
\maketitle
\tableofcontents

\chapter{Linux}
\section{Bash}
\verb|C^r (Control+r)| ricerca ricorsiva dei comandi usati in precedenza.\\

\noindent \verb|passwd| cambia la password dell'utente corrente\\

\noindent \verb|whoami| mostra l'utente corrente.
\section{Installazione}
empty
\newpage
\section{Package Manager}
\subsection{Debian-based (Aptitude)}
\noindent Aggiornare i repository:

\verb|sudo apt-get update|\\

\noindent Aggiornare i pacchetti installati:

\verb|sudo apt-get upgrade|\\

\noindent Installare un nuovo pacchetto:

\verb|sudo apt-get install *nome pacchetto*|\\

\noindent Cercare un pacchetto nei nostri repository:

\verb|apt-cache search *keyword*|\\

\noindent Rimuovere un pacchetto installato (mantiene i file di configurazione):

\verb|sudo apt-get remove *nome pacchetto*|\\

\noindent Rimuovere un pacchetto installato E i file di configurazione:

\verb|sudo apt-get purge *nome pacchetto*|\\

\noindent Rimuovere dipendenze non più necessarie:

\verb|sudo apt-get autoremove|\\

\noindent 
\newpage

\section{Controllo dei servizi}
\verb|systemctl| è il daemon responsabile del controllo dei servizi disponibili sulle macchine che usano systemd.\\

\noindent \verb|systemctl status *nome servizio*| mostra lo stato del servizio desiderato con le ultime righe del log.\\

\noindent \verb|sudo systemctl restart/stop *nome servizio*| riavvia/ferma il servizio desiderato.\\

\noindent \verb|sudo systemctl enable/disable *nome servizio*| abilita o disabilita il servizio specificato.



\newpage
\section{Editor di testo}
\subsection{VIM}
Aprire un file esistente (o crearlo se non esiste):

\verb|vim *nome file*|\\

\noindent Comandi più usati

\verb|a| per iniziare a scrivere DOPO il cursore:

\verb|i| per iniziare a scrivere SUL cursore:

\verb|A (shift+a)| per iniziare a scrivere alla fine della frase su cui si trova il cursore

\verb|gg| per spostare il cursore alla prima riga del file

\verb|G (shift+g)| per spostare il cursore all'ultima riga del file

\verb|:q| esci senza salvare (verrà chiesto di scrivere \verb|:q!| per uscire senza salvare se si è modificato il file)

\verb|:w| salva il file/le modifiche

\verb|:wq| salva il file/le modifiche ed esci

\verb|:term| divide lo schermo orizzontalmente e apre un terminale. \verb|C^W w (control+w w)| per passare dal terminale a VIM e viceversa. \verb|C^d| per chiudere la schermata su cui siamo (tipo il terminale).

\subsubsection{.vimrc}
E' il file che mantiene le impostazioni da utilizzare globalmente su VIM ogni volta che si apre un file. Si deve creare nella propria home directory.\\

Impostazioni utili da usare nel .vimrc:\\
\verb|set number| per  mostrare il numero della riga a lato.\\
\verb||


\chapter{title}
\chapter{title}

\end{document}          
